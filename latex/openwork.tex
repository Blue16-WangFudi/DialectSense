\documentclass[12pt]{article}
\usepackage{openwork}

\title{DialectSense: A Data-Centric WavLM Embedding Pipeline for Coarse Dialect Identification}
\author[1]{Author One}
\affil[1]{Affiliation One}
\author[2]{Author Two}
\affil[2]{Affiliation Two}
\date{}

% Convenience macros for the curated three-run artifact snapshots under latex/artifacts/.
\newcommand{\RunOne}{v1}
\newcommand{\RunTwo}{v2}
\newcommand{\RunThree}{v3}
\newcommand{\runfig}[2]{artifacts/#1/figures/#2}
\newcommand{\runfile}[2]{\nolinkurl{artifacts/#1/#2}}

\begin{document}

\maketitle

\begin{abstract}
\emph{DialectSense} is an end-to-end, reproducible dialect identification system built as a data-centric pipeline: audio quality control (QC) and preprocessing, WavLM-Large utterance embeddings, speaker-disjoint train/val/test splits, train-only label coarsening into coarse clusters, and supervised coarse-label classification.
The goal is robust \emph{coarse} dialect identification on a crowdsourced dataset where fine-grained province labels are imbalanced and some labels have too few speakers for a strict speaker-disjoint split.

Using the Xiangyin dialect corpus, our preprocessing stage reads 27{,}146 metadata rows, drops 49 rows missing province labels, and retains 20{,}800 clips after silence trimming and a minimum effective duration constraint.
We then prune labels that cannot satisfy the speaker-disjoint split constraints and train/evaluate on 27 province labels grouped into $K{=}12$ coarse clusters.
Across three experiment runs on a fixed 1{,}777-clip test set -- (v1) default MLP classifier, (v2) tuned LinearSVC, and (v3) stacked SVM+MLP with a meta LogisticRegression -- the best run reaches 0.408 accuracy and 0.230 macro-F1 on coarse cluster prediction.

Beyond offline evaluation, the repository provides a Makefile-driven pipeline and a Gradio Web UI for real-time chunk-level inference with per-cluster confidence curves, supporting a clean deliverable showcase and reproducible experimentation.
\par\smallskip
\noindent\textbf{Keywords:} dialect identification, WavLM, speaker-disjoint split, label coarsening, stacked classifier, reproducibility, real-time demo
\end{abstract}

\section{Introduction}

Dialect identification is a practical prerequisite for many speech applications, including dialect-aware ASR, regionalized TTS, and linguistic analysis of large audio collections.
In real-world crowdsourced data, however, province-level labels are often \emph{highly imbalanced} and some labels have too few speakers to support a strict speaker-disjoint evaluation split.
This repository takes a \emph{data-centric} approach: instead of training an end-to-end neural network from scratch, we focus on robust preprocessing, strong frozen speech representations, careful splitting, and transparent reporting artifacts.

\subsection{What This Repo Implements}

DialectSense implements the following end-to-end pipeline (also summarized in \autoref{fig:overview}):
\begin{enumerate}
    \item \textbf{Audio QC + preprocessing} (robust decode, silence trimming, RMS normalization).
    \item \textbf{WavLM-Large embeddings} as utterance-level representations.
    \item \textbf{Speaker-disjoint split} by \texttt{uploader\_id} (train/val/test).
    \item \textbf{Train-only label coarsening} into $K{=}12$ coarse clusters (KMeans on label centroids).
    \item \textbf{Coarse classifier training} and \textbf{evaluation} with standard metrics and visualizations.
    \item \textbf{Reporting + UI demo}: a Gradio UI with real-time chunk inference and confidence curves.
\end{enumerate}

\begin{figure}[hbt!]
    \centering
    \includegraphics[width=\linewidth]{figure/dialectsense_framework.jpg}
    \caption{DialectSense end-to-end pipeline implemented in this repository: preprocess $\rightarrow$ embed $\rightarrow$ split $\rightarrow$ coarsen $\rightarrow$ train $\rightarrow$ eval/report $\rightarrow$ UI.}
    \label{fig:overview}
\end{figure}

\subsection{Contributions and Document Roadmap}

This report is written in an academic-paper style while remaining faithful to the repository artifacts. The main contributions are:
\begin{itemize}
    \item \textbf{End-to-end, reproducible pipeline:} each stage is scriptable and produces explicit artifacts under \texttt{artifacts/<run\_name>/}.
    \item \textbf{Three-run comparison:} v1--v3 are compared using stored metrics and figures under \texttt{latex/artifacts/}.
    \item \textbf{Operational demo:} a Web UI supports streaming inference and visualization for a deliverable showcase.
\end{itemize}

The rest of this paper is organized as: related work (\S\ref{sec:related}), method (\S\ref{sec:method}), implementation (\S\ref{sec:implementation}), dataset (\S\ref{sec:dataset}), experiments (\S\ref{sec:experiments}), results (\S\ref{sec:results}), discussion (\S\ref{sec:discussion}), and conclusion (\S\ref{sec:conclusion}).

\section{Related Work}
\label{sec:related}

Dialect identification is commonly approached with acoustic feature extraction followed by supervised classification.
Modern systems often replace hand-crafted features with self-supervised speech representations, where a large pre-trained model serves as a general-purpose encoder and downstream tasks are solved with lightweight classifiers.
In this project, we use WavLM-Large as a frozen encoder and focus on the engineering aspects that determine reproducibility and robustness on crowdsourced data: audio quality control, speaker-disjoint splitting, and transparent evaluation artifacts.

This repository is intentionally empirical and artifact-driven. We therefore keep related work brief and do not introduce additional citations beyond the models and libraries directly used in the implementation.

\section{Method: Pipeline and System Design}
\label{sec:method}

DialectSense is organized as a sequence of deterministic pipeline stages, each producing explicit artifacts under \texttt{artifacts/<run\_name>/} (or the curated snapshots under \texttt{latex/artifacts/} used in this document).
This section describes the core method components and how they are implemented in the repository.

\subsection{Speech Representation: WavLM-Large Utterance Embeddings}

We use WavLM-Large as a frozen feature extractor.
Each preprocessed waveform is mapped to an utterance embedding by:
\begin{enumerate}
    \item computing hidden states with \texttt{output\_hidden\_states=True},
    \item averaging a mid-layer range (default: layers 6--12),
    \item mean-pooling over time to obtain a fixed-dimensional vector, and
    \item L2-normalizing the final embedding.
\end{enumerate}

Long utterances are handled by chunked embedding:
the audio is split into overlapping windows (default: 3.0~s chunks with 1.5~s hop, capped by \texttt{embed.chunk.max\_sec} and \texttt{embed.chunk.max\_chunks}), embedded per chunk, and aggregated (default: mean).

\subsection{Speaker-Disjoint Splitting with Label Constraints}

To avoid speaker leakage, splits are constructed by grouping on \texttt{uploader\_id}.
Because crowd data contains many low-frequency labels, the pipeline first prunes labels that cannot satisfy minimum sample and minimum speaker counts per split, then retries group splitting; if a feasible split still cannot be found, it iteratively drops the rarest remaining label until the constraints are satisfied.
The exact resulting split used in v1--v3 is stored in \runfile{\RunThree}{splits.csv}.

\subsection{Train-Only Coarse Label Mapping}

Rather than predicting dozens of imbalanced fine labels directly, DialectSense evaluates a \emph{coarse} classification task:
original province labels are mapped into $K{=}12$ clusters computed \emph{only from the training split} to avoid leakage.

The coarsening algorithm is:
\begin{enumerate}
    \item For each original label \(y\), compute a \emph{label centroid} embedding \(\mu_y\) by averaging the embeddings of training samples with label \(y\) (optionally with trimmed/weighted variants).
    \item L2-normalize each \(\mu_y\).
    \item Run KMeans with \(K{=}12\) on the set of label centroids \(\{\mu_y\}\).
    \item Assign every sample with label \(y\) to its cluster \(\mathrm{cluster}(y)\).
\end{enumerate}

Figure~\ref{fig:cluster_structure} visualizes the coarse cluster geometry (centroid cosine distances) and the 2D projection of label centroids.

\begin{figure}[hbt!]
    \centering
    \begin{tblr}{colsep=2pt, colspec={cc}}
        \includegraphics[width=0.475\linewidth]{\runfig{\RunThree}{cluster_centroid_cosine_distance.png}} &
        \includegraphics[width=0.475\linewidth]{\runfig{\RunThree}{label_centroids_2d.png}} \\
    \end{tblr}
    \caption{Coarse cluster structure for v3: cosine distances between cluster centroids (left) and a 2D projection of label centroids colored by coarse cluster (right).}
    \label{fig:cluster_structure}
\end{figure}

\subsection{Embedding Space Visualization}

To make representation quality and cluster overlap visible, the pipeline also exports 2D projections (UMAP if available, else t-SNE) of train and test embeddings colored by coarse cluster:

\begin{figure}[hbt!]
    \centering
    \begin{tblr}{colsep=2pt, colspec={cc}}
        \includegraphics[width=0.475\linewidth]{\runfig{\RunThree}{embedding_2d_train.png}} &
        \includegraphics[width=0.475\linewidth]{\runfig{\RunThree}{embedding_2d_test.png}} \\
    \end{tblr}
    \caption{Train/test embedding projections for v3 (colored by coarse cluster). These plots help diagnose overlap between clusters and potential sources of confusion.}
    \label{fig:emb2d}
\end{figure}

\subsection{Coarse Classifiers}

Given an utterance embedding \(x \in \mathbb{R}^d\), the coarse classifier outputs a cluster probability vector \(p(\cdot \mid x)\).
This repository supports several \texttt{scikit-learn} classifier families through configuration, including:
\begin{itemize}
    \item MLP (baseline),
    \item LinearSVC / LogisticRegression (fast linear baselines),
    \item Prototype-based cosine classifiers, and
    \item A stacked classifier (SVM + MLP $\rightarrow$ meta LogisticRegression) for improved accuracy.
\end{itemize}
The experiment section (\S\ref{sec:experiments}) compares the three concrete runs v1--v3 using stored metrics.

\subsection{Reporting and Artifacts}

Each run writes machine-readable metrics (\texttt{report\_coarse.json}) plus visual artifacts under \texttt{figures/} (confusion matrix, F1 by cluster, top confusion pairs, duration histograms, and 2D projections).
These artifacts are used directly in this document to avoid retyping and to keep the report evidence-driven.

\FloatBarrier

\section{Implementation}
\label{sec:implementation}

\subsection{Repository Structure}

The codebase is organized around a pipeline-first design:
\begin{itemize}
    \item \texttt{dialectsense/}: core library implementing all pipeline stages and the Web UI.
    \item \texttt{configs/}: JSON configuration files (smoke and full settings).
    \item \texttt{datasets/}: dataset snapshot (metadata and audio files).
    \item \texttt{artifacts/}: generated outputs, one folder per \texttt{run\_name}.
\end{itemize}

\subsection{Entrypoints}

The recommended interface is the Makefile:
\begin{lstlisting}
make smoke
make preprocess embed split coarsen train eval report CONFIG=configs/full.json
make ui CONFIG=configs/full.json
\end{lstlisting}

Each stage is also exposed as a CLI subcommand:
\begin{lstlisting}
python3 -m dialectsense.cli preprocess --config configs/full.json
python3 -m dialectsense.cli embed      --config configs/full.json
python3 -m dialectsense.cli split      --config configs/full.json
python3 -m dialectsense.cli coarsen    --config configs/full.json
python3 -m dialectsense.cli train      --config configs/full.json
python3 -m dialectsense.cli eval       --config configs/full.json
python3 -m dialectsense.cli report     --config configs/full.json
python3 -m dialectsense.cli ui         --config configs/full.json
\end{lstlisting}

\subsection{Configuration and Reproducibility}

All behavior is controlled by a JSON config file. Key fields include:
\begin{itemize}
    \item \texttt{data.*}: paths, inclusion filters, and subsampling.
    \item \texttt{audio.*}: preprocessing, QC rules, and duration thresholds.
    \item \texttt{embed.*}: WavLM settings and embedding chunking.
    \item \texttt{split.*}: speaker-disjoint split constraints and random seeds.
    \item \texttt{coarse.*}: KMeans settings for train-only label coarsening.
    \item \texttt{model.*}: coarse classifier family and hyperparameters.
\end{itemize}

For reproducibility, each run writes a resolved configuration snapshot to
\nolinkurl{artifacts/<run_name>/config.resolved.json}.

\subsection{Artifacts Produced by Each Stage}

The pipeline produces a complete set of machine-readable outputs and figures:
\begin{itemize}
    \item \texttt{audio\_qc.csv}: per-clip preprocessing outcomes and QC decisions.
    \item \texttt{splits.csv}: speaker-disjoint train/val/test splits.
    \item \texttt{label\_to\_cluster.json}, \texttt{cluster\_summary.md}: coarse mapping (train-only).
    \item \texttt{models/coarse\_model.joblib}: trained classifier bundle.
    \item \texttt{report\_coarse.json}, \texttt{top\_confusions.csv}: evaluation metrics and dominant errors.
    \item \texttt{figures/*.png}: diagnostics for QC, embeddings, and performance.
\end{itemize}

\subsection{Web UI (Streaming Inference)}

The Web UI provides a real-time demo mode:
\begin{itemize}
    \item microphone audio is streamed into a fixed-length chunker;
    \item each chunk is embedded with WavLM and classified into coarse clusters;
    \item the UI visualizes per-cluster confidence curves over time.
\end{itemize}

TODO: Add a screenshot file for the UI (for example, \texttt{latex/figure/ui\_realtime.png}) and include it as a figure.

\FloatBarrier

\section{Dataset}
\label{sec:dataset}

\subsection{Source, Files, and License}

The project is built around a locally stored snapshot of the Xiangyin dataset.
Relevant files in this repository are:
\begin{itemize}
    \item \texttt{datasets/metadata.csv}: metadata table (clip id, label, uploader id, and other fields).
    \item \texttt{datasets/oggs/}: audio files named as \texttt{<id>.ogg}.
    \item \texttt{datasets/LICENSE}: CC BY-NC-SA 4.0 license text.
\end{itemize}

\subsection{Task Definition}

Each example is an audio clip with a province-level label stored in the metadata column \texttt{dialect.province}.
We evaluate \textbf{coarse dialect identification}: original province labels are mapped into $K{=}12$ coarse clusters computed from training data only, and the classifier predicts the cluster id in $\{0,\dots,11\}$.

\subsection{Evidence-Backed Data Accounting}

Table~\ref{tab:data_summary} reports the exact counts used in our experiments, taken from the stored pipeline stats of v3.

\begin{table*}[t]
    \centering
    \small
    \begin{tblr}{
        width=\linewidth,
        colspec={X[l] Q[r] X[l]},
        row{1}={font=\bfseries},
    }
        Stage & Count & Evidence (file and key) \\
        \toprule
        Metadata rows & 27{,}146 & \runfile{\RunThree}{preprocess_stats.json}\linebreak\nolinkurl{read.n_metadata_rows} \\
        Missing province labels & 49 & \runfile{\RunThree}{preprocess_stats.json}\linebreak\nolinkurl{read.n_missing_label} \\
        Labeled clips considered & 27{,}097 & \runfile{\RunThree}{preprocess_stats.json}\linebreak\nolinkurl{votes_filter.n_in} \\
        Kept after audio QC & 20{,}800 & \runfile{\RunThree}{preprocess_stats.json}\linebreak\nolinkurl{audio_qc.n_kept} \\
        Dropped after trim (too short) & 6{,}297 & \runfile{\RunThree}{preprocess_stats.json}\linebreak\nolinkurl{audio_qc.drop_reasons} \\
        Kept labels after QC & 89 & \runfile{\RunThree}{split_stats.json}\linebreak\nolinkurl{prune_impossible_for_split.n_labels_in} \\
        Labels after split-feasibility pruning & 32 & \runfile{\RunThree}{split_stats.json}\linebreak\nolinkurl{prune_impossible_for_split.n_labels_out} \\
        Final labels used (province-level) & 27 & \runfile{\RunThree}{split_stats.json}\linebreak\nolinkurl{split.counts.*} \\
        Final clips used (train+val+test) & 20{,}079 & \runfile{\RunThree}{split_stats.json}\linebreak\nolinkurl{split.counts.*} \\
        \bottomrule
    \end{tblr}
    \caption{Dataset accounting for the v3 pipeline (all values are read from stored stats files).}
    \label{tab:data_summary}
\end{table*}

\subsection{Audio QC and Preprocessing}

Audio preprocessing is deterministic and conservative:
\begin{itemize}
    \item robust decoding and silence trimming via \texttt{ffmpeg} \texttt{silenceremove};
    \item resampling to 16~kHz mono;
    \item RMS normalization to \(-20\) dBFS with peak limiting (enabled by default);
    \item dropping clips whose effective duration after trimming is below 1.0 seconds.
\end{itemize}

\begin{figure}[t]
    \centering
    \begin{subfigure}{0.49\linewidth}
        \centering
        \includegraphics[width=\linewidth]{\runfig{\RunThree}{dropped_reasons.png}}
        \caption{Drop reasons}
    \end{subfigure}\hfill
    \begin{subfigure}{0.49\linewidth}
        \centering
        \includegraphics[width=\linewidth]{\runfig{\RunThree}{effective_duration_hist_kept.png}}
        \caption{Kept duration histogram}
    \end{subfigure}
    \caption{Audio QC outcomes for v3.}
    \label{fig:qc}
\end{figure}

\subsection{Splits and Speaker Disjointness}

The pipeline constructs speaker-disjoint train/val/test splits using the uploader identifier \texttt{uploader\_id}.
The v1--v3 comparisons share the same split file \runfile{\RunThree}{splits.csv}. Table~\ref{tab:splits} summarizes its sizes.

\begin{table}[t]
    \centering
    \small
    \begin{tblr}{colspec={l Q[r] Q[r]}, row{1}={font=\bfseries}}
        Split & Clips & Speakers \\
        \toprule
        Train & 16{,}258 & 5{,}261 \\
        Val & 2{,}044 & 658 \\
        Test & 1{,}777 & 658 \\
        \bottomrule
    \end{tblr}
    \caption{Speaker-disjoint split statistics computed from \runfile{\RunThree}{splits.csv}.}
    \label{tab:splits}
\end{table}

\begin{figure}[t]
    \centering
    \begin{subfigure}{0.49\linewidth}
        \centering
        \includegraphics[width=\linewidth]{\runfig{\RunThree}{cluster_counts.png}}
        \caption{Coarse cluster counts}
    \end{subfigure}\hfill
    \begin{subfigure}{0.49\linewidth}
        \centering
        \includegraphics[width=\linewidth]{\runfig{\RunThree}{speakers_per_split.png}}
        \caption{Speakers per split}
    \end{subfigure}
    \caption{Distribution diagnostics exported by the pipeline for v3.}
    \label{fig:data_dist}
\end{figure}

\FloatBarrier

\section{Experiments}
\label{sec:experiments}

\subsection{Evaluation Protocol}

All runs follow the same protocol:
\begin{itemize}
    \item perform audio QC and preprocessing;
    \item extract utterance embeddings using WavLM-Large;
    \item create speaker-disjoint train/val/test splits using \texttt{uploader\_id};
    \item compute a train-only coarse mapping into $K{=}12$ clusters;
    \item train a coarse classifier on the training split (with optional tuning);
    \item evaluate on the fixed test split.
\end{itemize}

The three-run comparison uses the same test set size, \(n_{test} = 1777\), as stored in each run report.

\subsection{Metrics}

We report:
\begin{itemize}
    \item \textbf{Accuracy} on coarse cluster prediction.
    \item \textbf{Macro-F1} across clusters (equal weight per cluster).
\end{itemize}
Per-cluster precision/recall/F1 and confusion matrices are included as diagnostics.

\subsection{Three Runs}

Each run is backed by a resolved config snapshot under \texttt{latex/artifacts/}.
Table~\ref{tab:run_defs} summarizes the key classifier differences.

\begin{table}[t]
    \centering
    \small
    \begin{tblr}{
        width=\linewidth,
        colspec={l X[l]},
        row{1}={font=\bfseries},
    }
        Run & Definition (evidence) \\
        \toprule
        v1 & Baseline configuration; see \runfile{\RunOne}{config.resolved.json}. \\
        v2 & LinearSVC tuned for validation accuracy; see \runfile{\RunTwo}{config.resolved.json}. \\
        v3 & Stacked classifier (SVM + MLP, meta LogisticRegression); see \runfile{\RunThree}{config.resolved.json}. \\
        \bottomrule
    \end{tblr}
    \caption{Three experiment runs compared in this paper.}
    \label{tab:run_defs}
\end{table}

\FloatBarrier


\section{Results}
\label{sec:results}

\subsection{Main Metrics}

Table~\ref{tab:main_results} reports accuracy and macro-F1 on the fixed test split (\(n_{test}=1777\)).
All values are read from the stored reports:
\runfile{\RunOne}{report_coarse.json},
\runfile{\RunTwo}{report_coarse.json}, and
\runfile{\RunThree}{report_coarse.json}.

\begin{table}[t]
    \centering
    \small
    \begin{tblr}{
        width=\linewidth,
        colspec={l l Q[r] Q[r]},
        row{1}={font=\bfseries},
    }
        Run & Model & Accuracy & Macro-F1 \\
        \toprule
        v1 & MLP baseline & 0.3911 & 0.2196 \\
        v2 & LinearSVC (tuned) & 0.3860 & 0.2654 \\
        v3 & Stacked (SVM + MLP + meta LR) & 0.4080 & 0.2298 \\
        \bottomrule
    \end{tblr}
    \caption{Main results for coarse cluster prediction.}
    \label{tab:main_results}
\end{table}

\subsection{Confusion Matrices}

Figure~\ref{fig:cms} compares row-normalized confusion matrices across the three runs.
To keep figures readable in a one-column layout, we show v1 and v2 side-by-side and v3 at full width.

\begin{figure}[t]
    \centering
    \begin{subfigure}{0.49\linewidth}
        \centering
        \includegraphics[width=\linewidth]{\runfig{\RunOne}{confusion_matrix_coarse.png}}
        \caption{v1: MLP}
    \end{subfigure}\hfill
    \begin{subfigure}{0.49\linewidth}
        \centering
        \includegraphics[width=\linewidth]{\runfig{\RunTwo}{confusion_matrix_coarse.png}}
        \caption{v2: tuned LinearSVC}
    \end{subfigure}

    \vspace{6pt}
    \begin{subfigure}{\linewidth}
        \centering
        \includegraphics[width=\linewidth]{\runfig{\RunThree}{confusion_matrix_coarse.png}}
        \caption{v3: stacked classifier}
    \end{subfigure}
    \caption{Row-normalized confusion matrices on the same test set.}
    \label{fig:cms}
\end{figure}

\subsection{Diagnostics: Per-Cluster F1 and Dominant Confusions}

Figure~\ref{fig:perf_v3} summarizes cluster-level diagnostics for v3: per-cluster F1 and the most frequent confusion pairs by count.
The underlying data are exported by the pipeline as \runfile{\RunThree}{report_coarse.json} and \runfile{\RunThree}{top_confusions.csv}.

\begin{figure}[t]
    \centering
    \begin{subfigure}{0.49\linewidth}
        \centering
        \includegraphics[width=\linewidth]{\runfig{\RunThree}{f1_by_cluster.png}}
        \caption{Per-cluster F1}
    \end{subfigure}\hfill
    \begin{subfigure}{0.49\linewidth}
        \centering
        \includegraphics[width=\linewidth]{\runfig{\RunThree}{top_confusions.png}}
        \caption{Top confusion pairs}
    \end{subfigure}
    \caption{Diagnostics exported for v3.}
    \label{fig:perf_v3}
\end{figure}

\FloatBarrier


\section{Discussion, Limitations, and Future Work}
\label{sec:discussion}

\subsection{What the Results Suggest}

\textbf{Accuracy vs. macro-F1 trade-off.}
Across the three runs, v3 improves accuracy over v1, while v2 achieves the best macro-F1 (Table~\ref{tab:main_results}).
Because macro-F1 weights each cluster equally, it is more sensitive to small-support clusters (see Appendix \ref{sec:appendix} for cluster supports and per-cluster F1).

\textbf{Dominant confusions concentrate in a few cluster pairs.}
The top confusion pairs for v3 (Figure~\ref{fig:perf_v3} and \runfile{\RunThree}{top_confusions.csv}) show repeated errors such as \(4 \rightarrow 11\), \(4 \rightarrow 0\), and \(2 \rightarrow 11\).
Interpreting these errors requires mapping clusters back to their province sets (Table~\ref{tab:cluster_map}), which reveals that some confusions occur between geographically/linguistically adjacent province groups (e.g., clusters containing provinces in Central/Eastern China).

\textbf{Some clusters remain unresolved.}
In v3, several clusters have F1=0 (Appendix Table~\ref{tab:per_cluster_metrics_v3}), indicating either severe overlap in the embedding space or insufficient training signal for those clusters under the current imbalance.

\subsection{Threats to Validity}

\begin{itemize}
    \item \textbf{Label noise and heterogeneous label space.} The upstream dataset includes many labels beyond mainland provinces (e.g., foreign regions or city-level labels). The pipeline prunes labels to satisfy split constraints, which changes the effective label set (Table~\ref{tab:data_summary}).
    \item \textbf{Imbalance.} Cluster sizes vary widely (Figure~\ref{fig:data_dist} and Table~\ref{tab:cluster_map}), and the test supports for minority clusters can be small, amplifying variance in per-cluster metrics.
    \item \textbf{Frozen representation.} WavLM embeddings are used without end-to-end fine-tuning; this improves engineering simplicity but may cap performance.
\end{itemize}

\subsection{Future Work (Grounded in Current Implementation)}

The current codebase already provides several levers for systematic improvement:
\begin{itemize}
    \item \textbf{Improve coarsening robustness:} \texttt{coarse.centroid.method} supports alternatives (e.g., trimmed mean or QC-weighted mean) that may reduce centroid noise for small labels.
    \item \textbf{Expand tuning beyond linear models:} v2 demonstrates val-driven tuning for accuracy; extending this to stacked components and regularization may improve both accuracy and macro-F1.
    \item \textbf{Better imbalance handling:} add explicit reweighting or resampling strategies at the classifier level, and report calibrated probabilities for UI confidence curves.
    \item \textbf{Add run-time provenance:} export per-stage timing and hardware metadata (TODO) to support reproducibility claims about speed/cost.
\end{itemize}


\section{Conclusion}
\label{sec:conclusion}

This paper presented an evidence-driven dialect identification pipeline centered on robust data processing and frozen speech representations.
DialectSense combines audio QC, WavLM-Large embeddings, speaker-disjoint splitting, train-only label coarsening, and configurable coarse classifiers into a reproducible workflow.
Using stored artifacts from three runs (v1--v3), we reported coarse-cluster performance on a fixed test set and provided visual diagnostics suitable for a clean deliverable showcase.

\subsection{Future Work}

Immediate next steps are:
\begin{itemize}
    \item export run-time and hardware metadata to support stronger reproducibility claims;
    \item add a UI screenshot figure to the paper;
    \item systematically tune coarsening and classifier settings to improve both accuracy and macro-F1 under imbalance.
\end{itemize}



\appendix
\section{Appendix: Coarse Mapping and Detailed Metrics}
\label{sec:appendix}

\subsection{Coarse Cluster Mapping (Train-Only)}

Table~\ref{tab:cluster_map} is derived from \runfile{\RunThree}{cluster_summary.md} and summarizes the train-only KMeans coarsening into $K{=}12$ clusters.

\begin{table*}[hbt!]
    \centering
    \begin{tblr}{r l r}
        \toprule
        \textbf{Cluster} & \textbf{Original province labels} & \textbf{Train samples} \\
        \midrule
        0 & Jiangxi, Zhejiang, Hunan & 2{,}876 \\
        1 & Shanxi, Gansu & 701 \\
        2 & Jilin, Liaoning, Heilongjiang & 690 \\
        3 & Taiwan & 169 \\
        4 & Yunnan, Anhui, Jiangsu, Hubei, Shaanxi & 3{,}647 \\
        5 & Guangxi Zhuang Autonomous Region & 386 \\
        6 & Inner Mongolia Autonomous Region & 144 \\
        7 & Shanghai & 270 \\
        8 & Beijing, Tianjin & 371 \\
        9 & Sichuan, Guizhou, Chongqing & 1{,}768 \\
        10 & Guangdong, Fujian & 2{,}800 \\
        11 & Shandong, Hebei, Henan & 2{,}436 \\
        \bottomrule
    \end{tblr}
    \caption{Coarse clusters used for training and evaluation (computed from training split label centroids only).}
    \label{tab:cluster_map}
\end{table*}

\subsection{Per-Cluster Metrics for v3}

Table~\ref{tab:per_cluster_metrics_v3} reports the per-cluster precision/recall/F1 and supports for v3, taken directly from \runfile{\RunThree}{report_coarse.json}.

\begin{table*}[hbt!]
    \centering
    \begin{tblr}{r r r r r}
        \toprule
        \textbf{Cluster} & \textbf{Support} & \textbf{Precision} & \textbf{Recall} & \textbf{F1} \\
        \midrule
        0 & 306 & 0.3733 & 0.4575 & 0.4112 \\
        1 & 56 & 0.1333 & 0.0357 & 0.0563 \\
        2 & 113 & 0.3448 & 0.0885 & 0.1408 \\
        3 & 24 & 0.0000 & 0.0000 & 0.0000 \\
        4 & 422 & 0.4337 & 0.3720 & 0.4005 \\
        5 & 61 & 0.0000 & 0.0000 & 0.0000 \\
        6 & 26 & 0.0000 & 0.0000 & 0.0000 \\
        7 & 22 & 0.0000 & 0.0000 & 0.0000 \\
        8 & 69 & 0.5000 & 0.1594 & 0.2418 \\
        9 & 165 & 0.4713 & 0.4970 & 0.4838 \\
        10 & 256 & 0.5574 & 0.6445 & 0.5978 \\
        11 & 257 & 0.3251 & 0.6148 & 0.4253 \\
        \midrule
        \textbf{Macro avg} & 1{,}777 & 0.2616 & 0.2391 & 0.2298 \\
        \textbf{Weighted avg} & 1{,}777 & 0.3839 & 0.4080 & 0.3786 \\
        \bottomrule
    \end{tblr}
    \caption{Per-cluster metrics for v3 (stacked classifier).}
    \label{tab:per_cluster_metrics_v3}
\end{table*}

\subsection{Top Confusion Pairs for v3}

Table~\ref{tab:top_confusions_v3} lists the most frequent confusion pairs (true$\rightarrow$pred) for v3 from \runfile{\RunThree}{top_confusions.csv}.

\begin{table}[hbt!]
    \centering
    \begin{tblr}{r r r r}
        \toprule
        \textbf{True} & \textbf{Pred} & \textbf{Count} & \textbf{Rate} \\
        \midrule
        4 & 11 & 113 & 0.2678 \\
        4 & 0 & 95 & 0.2251 \\
        2 & 11 & 67 & 0.5929 \\
        0 & 4 & 65 & 0.2124 \\
        11 & 4 & 50 & 0.1946 \\
        10 & 0 & 48 & 0.1875 \\
        0 & 10 & 44 & 0.1438 \\
        0 & 11 & 36 & 0.1176 \\
        5 & 10 & 35 & 0.5738 \\
        8 & 11 & 33 & 0.4783 \\
        \bottomrule
    \end{tblr}
    \caption{Top confusion pairs for v3 (rate is normalized by the true-class row sum).}
    \label{tab:top_confusions_v3}
\end{table}

\subsection{Where to Find All Plots}

Each curated run snapshot contains a complete figure set under \nolinkurl{latex/artifacts/<run>/figures/} (e.g., \runfile{\RunThree}{figures/}).
These include duration histograms, 2D embedding plots, confusion matrices, and error breakdowns.


\section*{Acknowledgments}
TODO: Add team acknowledgments / course info if required by the deliverable template.

\clearpage
\section*{References}
\begingroup
\small
\setlength{\itemsep}{0.25em}
\begin{thebibliography}{9}

\bibitem{wavlm}
S. Chen et al.
\newblock WavLM: Large-Scale Self-Supervised Pre-Training for Full Stack Speech Processing.
\newblock \url{https://arxiv.org/abs/2110.13900}.

\bibitem{sklearn}
F. Pedregosa et al.
\newblock Scikit-learn: Machine Learning in Python.
\newblock \emph{Journal of Machine Learning Research}, 12:2825--2830, 2011.
\newblock \url{https://jmlr.org/papers/v12/pedregosa11a.html}.

\bibitem{umap}
L. McInnes, J. Healy, and J. Melville.
\newblock UMAP: Uniform Manifold Approximation and Projection for Dimension Reduction.
\newblock \url{https://arxiv.org/abs/1802.03426}.

\bibitem{xiangyin}
Xiangyin dataset repository.
\newblock \url{https://github.com/cxcxcxcx/xiangyin_dataset}.

\end{thebibliography}
\endgroup


\end{document}
